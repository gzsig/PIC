\begin{samepage}
    \section{Introdução} 
    \vspace{1cm}
    \subsection{Definição justificada dos objetivos e da sua relevância}
     A palavra computador é usada desde o século XVII, tendo a sua primeira referência escrita datada de 1613. No entanto, por muito tempo a palavra computador não tinha o mesmo significado que leva hoje. Até a década de 1940 a palavra "computador" era tida como uma profissão de alguém que calcula, segundo o dicionário Michaelis: “Aquele ou aquilo que calcula baseado em valores digitais; calculador, calculista” \cite{michaels:computador}.\par
    O primeiro computador digital eletrônico de grande escala, o ENIAC (Electrical Numerical Integrator and Calculator), foi criado em fevereiro de 1946 pelos cientistas norte-americanos John Presper Eckert e John W. Mauchly, da Electronic Control Company. No final de sua operação em 1956, o ENIAC continha 20.000 tubos de vácuo; 7.200 diodos de cristal; 1.500 relés; 70.000 resistores; 10.000 capacitores; e aproximadamente 5.000.000 de juntas soldadas à mão. Ele pesava mais de 27 toneladas, tinha aproximadamente 2,4m * 0,9m * 30m de tamanho, ocupava 167 m2 e consumia 150 kW de eletricidade. \cite{wazlawick:historia_da_computacao}\par
    Já, em 1969 o computador de bordo da Apollo 11, missão que levou o homem a lua, tinha 32.768 bits de RAM, o suficiente para armazenar um texto não formatado com cerca de 2.000 palavras, ou seja, o equivalente ao número de caracteres contido nesta proposta. Em 2018, o iPhone XS, com 4GB de RAM (ou 34.359.738.368 bits), tem cerca de 1 milhão de vezes mais memória que o Apollo Guinche Computer. \cite{uol:iphone_vs_appolo}
    
    \subsection{Metodologia a ser empregada}
    Para o desenvolvimento da entrega do protótipo do computador clássico, computador de 8-bits, e para o simulador do computador quântico web, a principal metodologia utilizada será Project Based Learnig (PBL). De acordo com (VAN ANDEL, 2019), o PBL envolve os alunos em um processo rigoroso de investigação, onde eles fazem perguntas, encontram recursos e aplicam informações para resolver problemas do mundo real. Assim, assumisse que esta é a melhor metodologia para desenvolver um protótipo físico de um computador.\par
	Para o desenvolvimento da maior parte do projeto de pesquisa, a principal metodologia utilizada, será de pesquisa bibliográfica, assim recuperando conhecimento científico acumulado sobre o assunto. Segundo (SASSO DE LIMA, 2007), o conhecimento da realidade não é apenas a simples transposição dessa realidade para o pensamento, pelo contrário, consiste na reflexão crítica que se dá a partir de um conhecimento acumulado e que irá gerar uma síntese, o concreto pensado.

\end{samepage}
\newpage