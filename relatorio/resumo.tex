\thispagestyle{plain}
\begin{center}
    \Large
    \textbf{ESTUDOS, IMPLEMENTAÇÕES E IMPACTOS SOCIAIS:}
    
    \vspace{0.2cm}
    \large
    ESTUDOS, IMPLEMENTAÇÕES E IMPACTOS SOCIAIS
    
    \vspace{0.2cm}
    \textbf{Gabriel R. Zsigmond}
    
    \vspace{0.4cm}
    \textbf{Resumo}
\end{center}
O presente projeto, "Computadores Clássicos e Quânticos: Estudos, Implementação, Simuladores e Impactos Sociais", se propõe a estudar a história do computador e sua evolução. A mais recente inovação na área é a computação quântica, que certamente, inaugura a próxima geração de computadores. A computação quântica muda toda uma arquitetura na forma de computação, permitindo que os novos computadores sejam exponencialmente mais eficientes quando comparados aos mais modernos da atualidade. O projeto se propõe a entender e prototipar um computador tradicional de 8 bits em hardware, usando apenas portas lógicas simples, a fim de ilustrar, de forma clara, o funcionamento de um computador tradicional. Também, esse projeto busca entender e estimar as consequências sociais que os avanços da tecnologia e o desenvolvimento da computação quântica pode gerar. Entende-se que para essa análise, se faz necessário, inicialmente, uma ampla revisão bibliográfica, a fim de comparar esses dois tipos de computadores. Para ilustrar esta, será desenvolvida uma aplicação web que simula um computador tradicional e um computador quântico executando o mesmo algoritmo. É esperado que ao final do projeto, esse estudo traga um amplo e aprofundado conhecimento da área, além de, uma contribuição em relação ao impacto social do uso computação da quântica.