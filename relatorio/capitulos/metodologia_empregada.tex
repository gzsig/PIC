\section{Metodologia empregada}
\label{methodologies}
Para a realização da presente pesquisa de iniciação científica, foi indispensável a realização de uma revisão bibliográfica, afim de recuperar e aprofundar o conhecimento científico já produzido, assim bem como, se interar do estado da arte na área. Segundo Telma Cristiane Sasso de Lima, o conhecimento da realidade não é apenas a simples transposição dessa realidade para o pensamento, mas sim a reflexão crítica, que se dá a partir de um conhecimento acumulado que irá gerar uma síntese, o concreto pensado \cite{1}. E também a utilização do processo científico para a elaboração e efetivação do projeto em si. Ambas metodologias citadas acima são cruciais para o desenvolvimento do relatório final na área de pesquisa em computação, já que em grande parte dos estudos, a utilização do processo científico é frequentemente utilizada para um maior entendimento da obra e para que a construção do projeto possa se tornar mais facilmente executável.

Assim, para o desenvolvimento da entrega do protótipo — computador de 8-bits — e para o simulador do computador quântico web, a principal metodologia utilizada será Project Based Learnig (PBL). De acordo com David Van Andel, o PBL envolve os alunos em um processo rigoroso de investigação, onde eles fazem perguntas, encontram recursos e aplicam informações para resolver problemas do mundo real \cite{3}. Entende-se que esta é a melhor metodologia para desenvolver um protótipo físico de um computador e programar um site.

Nos dois capítulos a seguir, \ref{classic_comp} e \ref{quantum_comp}, respectivamente Computação Clássica e Computação Quântica são apresentadas em uma curta revisão bibliográfica 

\newpage