\section{Computação Quântica} 
\label{quantum_comp}
Como proposto pelo cronograma desse estudo\footnote{página \pageref{updated_timeline} tabela \ref{updated_timeline}}, o presente capítulo, a ser desenvolvido no decorrer do segundo semestre, adentrará o desenvolvimento da Computação Quântica assim como a descrição de sua arquitetura – tema de extrema atualidade. O capítulo partirá do estudo de Alexander Holevo, publicado em 1973, que obteve como resultado o conhecido "teorema de Holevo"\footnote{um importante teorema limitativo na computação quântica, um campo interdisciplinar da física e da ciência da computação. Às vezes chamado de limite de Holevo, uma vez que estabelece um limite superior para a quantidade de informação que pode ser conhecida sobre um estado quântico} – marco da computação quântica –, perpassando pela sua historia até os dias atuais. Assim, poderemos compreender as implicações dos avanços em relação aos computadores quânticos, que como mencionado pelo artigo “Google claims its quantum computer can do the impossible in 200 seconds” \cite{16} já consegue resolver um problema, que levaria 10.000 anos para ser solucionado pelo supercomputador (clássico) mais rápido do mundo, em 200 segundos.
\newpage