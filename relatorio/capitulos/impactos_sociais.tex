\section{Impactos sociais}
\label{social_impacts}
O que os computadores quânticos podem fazer?  Eles podem facilmente fazer o que parece ser uma tarefa impossível para computadores clássicos. Eles conseguem resolver e quebrar algoritmos de criptografia que protegem os dados de um usuário e a infraestrutura da Internet.  

EX: Em outubro de 2019, o Google revelou sua bem-sucedida experiência ``Quantum Supremacy'' alegando que um problema particularmente difícil foi resolvido em 200 segundos \cite{15}.

A criptografia atual é baseada em matemática e decodificação e essa é uma tarefa bastante complicada para computadores clássicos, já os computadores quânticos a executam com grande facilidade. 

EX: O algoritmo de fatoração de Shor\footnote{Um algoritmo de computador quântico de tempo polinomial para fatoração de número inteiro. Informalmente, resolve o seguinte problema: Dado um número inteiro, encontre seus fatores primos. Foi inventado em 1994 pelo matemático americano Peter Shor} é de particular importância porque esse algoritmo quântico significa que a criptografia de chave pública pode ser facilmente quebrada caso o computador quântico seja suficientemente grande.
\newpage