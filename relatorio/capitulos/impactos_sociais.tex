\section{Impactos sociais}
\label{social_impacts}
O avanço da computação quântica seguramente afetará o mundo atual de diversas formas. Entre elas, impactando temáticas como: privacidade; política; otimização de logística e melhor tomada de decisão por máquina.

O principal motivo é resultado do desempenho mais ágil que computadores quânticos tem quando são comparados aos mais modernos da atualidade – clássicos – podem facilmente fazer o que parece ser uma tarefa impossível para computadores clássicos. Eles conseguem resolver e quebrar algoritmos de criptografia que protegem os dados de um usuário e a infraestrutura da Internet, além de otimizar melhores caminhos para logística e sempre tomarem a melhor decisão.

Em outubro de 2019, o Google revelou sua bem-sucedida experiência “Quantum Supremacy”, alegando que um problema particularmente difícil havia sido resolvido em 200 segundos 
 \cite{15}.

\subsection{Criptografia \& Privacidade}
A criptografia atual é baseada em matemática e decodificação, sendo essa é uma tarefa bastante complicada para computadores clássicos. No entanto,os computadores quânticos a executam com grande facilidade. O algoritmo de fatoração de Shor\footnote{Um algoritmo de computador quântico de tempo polinomial para fatoração de número inteiro. Informalmente, resolve o seguinte problema: Dado um número inteiro, encontre seus fatores primos. Foi inventado em 1994 pelo matemático americano Peter Shor} é de particular importância para essa temática, pois significa que a criptografia de chave pública pode ser facilmente quebrada. Tendo em vista a “falha” na criptografia, o tema de privacidade vem à tona, sendo abordado no atual capítulo como esse fenômeno se comporta em três situações: negócios privados, transações e política.

\subsubsection{Criptografia}
Muitos criptosistemas são construídos usando problemas matemáticos mais difíceis do que um computador clássico é capaz de resolver. No entanto, um computador quântico tem a capacidade computacional de encontrar soluções para os algoritmos criptográficos em uso hoje. Problemas criptográficos que usam fatoração são excelentes exemplos de problemas que podem ser resolvidos com um computador quântico, porque tanto a entrada quanto a saída são, cada uma, um único número. Observe que os números usados na chave são enormes, portanto, uma quantidade significativa de qubits é necessária para calcular o resultado. A capacidade de um computador quântico de resolver algoritmos criptográficos é um problema extremamente sério, e já existem empresas – como a Microsoft por exemplo – que já estão trabalhando em protocolos de criptografia segura quântica para substituir aqueles que serão vulneráveis a ataques quânticos.

\subsubsection{Negócios privados}
Empresas modernas de forma geral armazenam e gerenciam a maioria de suas informações pessoais e confidenciais de forma online, em uma nuvem com conexão ininterrupta à web. Assim, é quase impossível conduzir tais informações de forma a evitar que os dados caiam nas mãos de terceiros. Por  esse motivo a criptografia está sendo incorporada aos planos de segurança de dados em nuvem de empresas, mantendo seus dados privados e seguros, independentemente de sua localização. 

De acordo com o estudo “Data Breach” da IBM em 2018, o custo para cobrir uma violação de dados média é de US \$ 3,86 milhões. Estima-se que este custo deve aumentar 6,4\% em 2019, com a probabilidade de uma violação recorrente nos próximos dois anos, chegando a aumentar 27,9\%. Essa estatística por si só é suficiente para as empresas considerarem maneiras de otimizar suas medidas de segurança,que garantam a proteção ideal dos dados de suas empresas.

\subsubsection{Transações}
No modelo atual, as transações acontecem de forma criptografada. A melhor maneira de essa temática é através da internet. Qualquer endereço que começe com “https://” é criptografado, o que significa que ninguém, a não ser quem acessa a página e o servidor tem acesso aos conteúdos transacionados. Com alguns níveis de abstração, esse processo também acorre com transações bancarias, em que apenas o dono da conta consegue visualizar o saldo presente na conta e fazer operações. Já no modelo futuro, quando os computadores quânticos tiverem a estabilidade que os computadores clássicos tem hoje, todos esses dados e acessos serão públicos. Este é um pensamento assustador, porém muito realista.

Um dos métodos mais eficazes para proteger dados é por meio de criptografia. Assim como se tranca as portas de espaços importantes em empresas para impedir uma invasão, é preciso fazer o mesmo com os dados por meio de criptografia. Em vez de trancar uma porta física, a criptografia coloca algoritmos baseados em algoritmos matemáticos para protegê-los de hackers.


\subsubsection{Política}
Com o avanço da computação quântica, uma realidade de perda de privacidade se aproxima rapidamente. Isso afetará involuntariamente a política interna e externa de muitos países. 

No escopo interno, as implicações da perda da privacidade das comunicações são preocupações importantes para os direitos humanos, por exemplo, em países onde regimes políticos opressores encontram interesse em manter a ficção de que todos os sujeitos concordam com seus pontos de vista. Com esses avanços, será possível usar medidas de vigilância secretas para monitorar dissidentes e rastrear as atividades de pessoas consideradas suspeitas que poderia sufocar a dissidência política, um objetivo que tem sido aspirado muitas vezes historicamente, mas nunca antes alcançado.  Assim, é claramente uma tentação para os governos realizar esse tipo de vigilância. O problema potencial não está isolado dos regimes opressores, mas provavelmente aparecerá de uma forma diferente (ilegal) nas democracias tradicionais, onde os cidadãos tradicionalmente atribuem um alto valor à privacidade. As implicações são amplas e para as democracias menos seguras são consideravelmente mais sinistras.

No escopo externo,  há a questão da desigualdade de conhecimento – informações sigilosas poderão ser acessadas por outros países – e portanto, será possível aumentar o poder de barganha em meio as negociações de comercio e relações internacionais.

\subsection{Otimização}
A computação quântica permite a otimização em diversas áreas. Essas incluem busca de melhores caminhos – logística em tempo-hábil,  melhor tomada de decisão de forma não empírica, robótica – tomar SEMPRE a melhor decisão e vale pena enfrentar uma decisão computacional?

\subsubsection{Busca de melhores caminhos – logística em tempo-hábil}
O problema de encontrar o caminho mais curto entre dois pontos em um grá- fico ponderado é antigo. A questão de quais algoritmos clássicos podem ser acelerados pela computação quântica é, obviamente, muito interessante. No mo- mento, existem apenas algumas técnicas gerais conhecidas no campo da computação quântica e a descoberta de novos problemas, passíveis de acelerações quânticas, é uma alta prioridade. Uma dessas técnicas gerais de soluções de problemas por meio da computação quântica foi considerada no Trabalho de Heiligman  \cite{21}.

No caso clássico, o trabalho total para o o algoritmo é max $ (O(kn^2),  O(n^2)) = O(n^2) $  qual é minimizado tomando $ k = 1 $ , portanto $ W classico=O(n^2) $

No caso quântico, a situação é um pouco diferente. O trabalho total na primeira etapa é apenas o máximo de o trabalho nas próximas duas etapas, desde o trabalho na primeira etapa é sempre dominada por esses outros fatores de trabalho. O trabalho total é, portanto, max$ (O(k^1/2 n^3/2), O(k^1/2 n^2)) $

e para minimizar isso, o parâmetro $ k $ deve ser escolhido para tornar esses dois fatores de trabalho iguais. Definição $ k^1/2 n^3/2= k^-1/2 n^2 $ dá $ k = n^1/2 $ e portanto $W quantico = O(n^7/4) $

Na verdade, isso é uma melhoria em relação ao clássico fator de trabalho de $ n^2 $

\subsubsection{Machine Learning}
Em geral, os computadores quânticos não são desafiados pela quantidade de computação necessária para resolver um problema. Em vez disso, o desafio é obter um número limitado de respostas e restringir o tamanho das entradas. Por causa disso, os problemas de Machine Learning (ML) muitas vezes não funcionam perfeitamente devido à grande quantidade de dados de entrada. No entanto, os problemas de otimização são um tipo de problema de ML que podem ser adequado para um computador quântico.

Imagine que você tem uma grande fábrica e o objetivo é maximizar a produção. Para fazer isso, cada processo individual precisaria ser otimizado por conta própria, bem como comparado com o todo. As configurações possíveis de todos os processos que precisam ser considerados são exponencialmente maiores do que o tamanho dos dados de entrada. Com um espaço de busca exponencialmente maior do que os dados de entrada, problemas de otimização são viáveis para um computador quântico.
  
Além disso, devido aos requisitos exclusivos da programação quântica, um dos benefícios inesperados do desenvolvimento de algoritmos quânticos é a identificação de novos métodos para resolver problemas. Em muitos casos, esses novos métodos podem ser trazidos de volta à computação clássica, produzindo melhorias significativas. A implementação dessas novas técnicas é o que chamamos de algoritmos inspirados no quantum.

\subsubsection{Robótica – tomar SEMPRE a melhor decisão}
Nos dias atuais, decisões são tomadas de forma não otimizadas, e ainda, muitas vezes baseadas em “achismo”. Nesse aspecto a nova tecnologia baseada na física quântica promete grandes avanços. 


\subsubsection{Vale pena enfrentar uma decisão computacional?}

\subsection{Considerações finais - incompleto}
Assim, o impacto que o avanço da física quântica aplicada a computação terá na criptografia, supera as barreiras técnicas e afeta de maneira irreversível diversas outras áreas. Para mitigar os danos gerados por esse avanço, em paralelo estão sendo feitos estudos que avançam na área da criptografia quântica. Mas esses ainda não apresentam soluções aplicáveis e viáveis em tempo hábil.

\newpage