\section{Computação Clássica} 
Entende-se a importancia de se compreender a origem e o desenvolvimento da computação clássica para o desenrolar da pesquisa, o que será apresentado em meio a este cápitulo.

A computação clássica consiste em computadores que dependem da física clássica para operar. Estes são os computadores tradicionais que usamos em nosso dia-a-dia – seja eles Apple, Samsung, Dell ou qualquer outro –, também clássificados como computadores binários, pois processam as instruções a partir de numeros binários, compostos apenas pelos simbolos “1” e “0”, ligado e desligado respectivamente. Assim, julga-se importante e de larga relevância ao tema compreender essa representação numérica. 

Números binários ou números em base 2 são compostos por apenas dois dígitos, [0...1]. Dessa forma, seu funcionamento é similar ao sistema decimal, ou base 10, que são compostos por dez dígitos, [0...9]. No sistem decimal, é simples contar até nove, porém não existe um simbolo ou dígito para representar o número dez, sendo então representado  dois dígitos, “10”. Isto é uma simples lógica de posicionamento. Mais uma vez, após o número 99, é necessário ultilizar a mesma regra para representar o número cem, “100”. Em base 2, o número zero é representado pelo simbolo 0, e o número um por 1. O mesmo dilema é enfrentado ao chegar no próximo valor, dois. E então é usada a mesma lógica de posicionamento, em base dois. O número dois é representado por “10”, o três por “11”, quatro por “100” e assim por diante. Dessa forma, números binários podem se tornar longos e compostos por muitos dígitos. Em computação, esses dígitos são chamados de bits. \cite{6} É com base nos \label{bits}bits\footnote{A menor unidade de informação que pode ser armazenada ou transmitida na comunicação de dados.} ligados e desligados que o computador baseia sua linguagem. Para transforma-lo em base dez é preciso avaliar o valor de cada bit de acordo com a sua posição. 

Exemplo: número binário $1011$:

\[ 1011(b) = 1*2^3 + 0*2^2 + 1*2^1 + 1*2^0\]
\[ = 8 + 0 + 2 + 1 = 11(decimal)\]

O peso de cada bit de um número binário depende da sua posição relativa ao número completo, sempre partindo da direita para a esquerda.

\begin{itemize}
  \item O peso do primeiro bit é $bit * 2^0$
  \item O peso do segundo bit é $bit * 2^1$
  \item O peso do terceiro bit é $bit * 2^2$
  \item O peso do quarto bit é $bit * 2^3$
\end{itemize}

A formúla ilustrada acima, pode ser exemplificada em uma fórmula genérica: 

\[= nth\: bit * 2^{n-1}\]

É possível notar que a regra para números binários, se repete para números em base 10.

Exemplo: número decimal $4392$:
\begin{itemize}
  \item O peso do primeiro bit é $2 * 10^0$
  \item O peso do segundo bit é $9 * 10^1$
  \item O peso do terceiro bit é $3 * 10^2$
  \item O peso do quarto bit é $4 * 10^3$
\end{itemize}
\[ 4392 = 4*10^3 + 3*10^2 + 9*10^1 + 2*10^0\]
\[= nth\: bit * 10^{n-1}\]

Essa regra se mantem verdadeira para qualquer base númerica.

\[= nth\: bit * (base)^{n-1}\]

Ao decorrer do texto serão referidos números em base 2, 10 e 16. 

\subsection{A computação clássica e sua evolução}
\newpage
