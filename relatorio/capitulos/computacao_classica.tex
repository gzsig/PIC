\section{Computação Clássica} 
Computadores clássicos são todos aqueles que a maior parte da população mundial conhece, por exemplo computadores da Apple, Samsung, Dell dentre outros. É possível questionar o que todos esse tem em comum, a final todos possuem características bastante distintas. A resposta é bem simples, esses usam fenômenos da física clássica para operar. Assim, são classificados como computadores clássicos (ou binários), esses processam dois tipos de sinais, 1 e 0, ligado e desligado respectivamente. É com base nos \label{bits}bits\footnote{A menor unidade de informação que pode ser armazenada ou transmitida na comunicação de dados.} ligados e desligados que o computador baseia sua linguagem.

Esses computadores podem ser referidos como computadores binários, assim, julga-se importante e de larga relevância ao tema compreender o que é um número binário. 

Números binários ou números em base 2 são compostos por apenas dois dígitos, [0...1]. Assim, seu funcionamento é similar ao sistema decimal, ou base 10 que são compostos por dez dígitos, [0...9]. No sistem decimal, é simples contar até nove, porém não existe um simbolo ou digito para representar o número dez e então é representado por “10”, dois digitos. Isto é uma simples lógica de posicionamento. Mais uma vez, após o número 99, é necessário ultilizar a mesma regra para representar o número cem, “100”. Em base 2, o número zero é representado pelo simbolo 0, e o número um por 1. O mesmo dilema é enfrentado ao chegar no próximo valor, dois. E então é usada a mesma lógica de posicionamento, em base dois. O número dois é representado por “10”, o três por “11”, quatro por “100” e assim por diante. Dessa forma, números binários podem se tornar longos e compostos por muitos digitos. Em computação, esses digitos são chamados de bits. \cite{6} Para transforma-lo em base dez é preciso avaliar o valor de cada bit de acordo com a sua posição. 

Exemplo: número binário $1011$:

\[ 1011(b) = 1*2^3 + 0*2^2 + 1*2^1 + 1*2^0\]
\[ (decimal) = 8 + 0 + 2 + 1 = 11\]

Fórmula genética transformar de base 2 para base 10 é: $= nth\: bit * 2^{n-1}$ e a fórmula genética para qualquer base é: $= nth\: bit * (base)^{n-1}$

\subsection{A computação clássica e sua evolução}
\newpage
