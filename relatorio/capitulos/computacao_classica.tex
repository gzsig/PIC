\section{Computação Clássica} 
Computadores clássicos são todos aqueles que a maior parte da população mundial conhece, por exemplo computadores da Apple, Samsung, Dell dentre outros. É possível questionar o que todos esse tem em comum, a final todos possuem características bastante distintas. A resposta é bem simples, esses usam fenômenos da física clássica para operar. Assim, são classificados como computadores clássicos (ou binários), esses processam dois tipos de sinais, 1 e 0, ligado e desligado respectivamente. É com base nos \label{bits}bits\footnote{A menor unidade de informação que pode ser armazenada ou transmitida na comunicação de dados.} ligados e desligados que o computador baseia sua linguagem.

Esses computadores podem ser referidos como computadores binários, assim, julga-se importante e de larga relevância ao tema compreender o que é um número binário. 

Números binários usam base 2, por tanto, qualquer valor é composto por apenas dois digitos, 0 e 1. Em computação, esses digitos são chamados de bits. \cite{6}

\subsection{A computação clássica e sua evolução}
\newpage
