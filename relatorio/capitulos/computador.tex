\section{Computador} 
\textbf{Ressalva: }
\textit{ Esse cápitulo foi reservado para descrever o passo a passo de como desenvolver um computador clássico apenas com portas logicas. Devido ao fato que as compras solicitadas para o desenvolvimento dessa parte do projeto atrasou um tempo  bastante significante e devido a pandemia gerada pelo COVID-19 eu não estou morando em São Paulo. Seguindo o cronograma a montagem seria feita no decorrer dos primeiros meses do projeto, tendo em vista que o protótipo ainda não foi desenvolvido e a segundo metade do cronograma é inteiramente voltada a computação quântica devida a sua relevância ao projeto e complexidade, é proposto que essa parte seja adiada para um possível futuro trabalho.}
\newline
\newline
Construir um computador parece uma tarefa complicada e ousada. Porém, uma CPU\footnote{CPU é a sigla para Central Process Unit, ou Unidade Central de Processamento. É o principal item de hardware do computador, que também é conhecido como processador, essa é a parte responsável por calcular e realizar tarefas determinadas pelo usuário.} é bastante simples em operação depois que os fundamentos por trás de todos os seus processos são compreendidos. Desta maneira, este capítulo destina-se em explicar o passo a passo para que qualquer pessoa interessada, seja capaz de construir seu próprio computador e obter o conhecimento que acompanha o processo.

\subsection{Módulos}
Para facilitar a compreensão, e também o desenvolvimento do computador, este capítulo será dividido em alguns subcapítulos, em que cada qual abordará sobre uma parte do computador.

\subsubsection{Clock}
O clock do computador é uma parte essencial para o seu funcionamento. Este tem a função de sincronizar todas as operações. A ação mais rápida que o computador consegue executar é equivalente a uma vibração do seu clock.

\subsubsection{Registers}
A maioria das CPUs possuem vários registradores que armazenam pequenas quantidades de dados processados pela CPU. Em nossa CPU de breadboard, criaremos três registradores de 8 bits: A, B e IR. Os registradores A e B são para uso geral. Já o IR (instruction register), apesar de funcionar da mesma forma, é usado para armazenar a instrução atual que está sendo executada.

\subsubsection{Arithmetic logic unit (ALU)}
A parte da unidade lógica aritmética (ALU) de uma CPU geralmente é capaz de executar várias operações aritméticas, bit a bit e de comparação em números binários. Em nossa CPU de breadboard, a ALU pode apenas adicionar e subtrair. Ele está conectado aos registradores A e B e gera a soma de A + B ou a diferença de A-B.

\subsubsection{Random access memory (RAM)}
A memória de acesso aleatório (RAM) armazena o programa que o computador está executando, bem como todos os dados que o programa precisa. Nosso computador de breadboard utiliza endereços de 4 bits, o que significa que ele terá apenas 16 bytes de RAM, limitando o tamanho e a complexidade dos programas que poderá executar.

\subsubsection{Program counter}
O contador do programa (Program counter) conta em binário para acompanhar qual instrução o computador está executando no momento.

\subsubsection{Output register}
O registrador de saída é semelhante a qualquer outro registrador (como os registradores A e B), exceto que, em vez de exibir seu conteúdo em binário em 8 LEDs, ele exibe seu conteúdo de forma decimal em um display de 7 segmentos, o que requer uma lógica complexa.

\subsubsection{CPU control logic}
A lógica de controle é o coração da CPU. É o que define os códigos de operação (opcode) que o processador reconhece e o que acontece quando ele executa cada instrução.

\subsubsection{Materiais Necessários}

\newpage