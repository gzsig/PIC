\section{Criptografia}
Conforme definido por Bruce Schneier \textit{``The art and science of keeping messages secure is cryptography […].''} \cite{13} Embora a criptografia seja considerada fundamental em nossas vidas digitais, ela não está especificamente relacionada à computação. A criptografia existe em diversas formas há milênios.

Na segurança cibernética, há uma série de preocupações quando se trata de dados. Isso inclui confidencialidade, integridade, disponibilidade e não repúdio.

\textbf{A confidencialidade} significa que nossos dados não podem ser acessados / lidos por usuários não autorizados.

\textbf{A integridade} do dado diz respeito à originalidade(?) na qual os dados chegam à nós, estando 100\% intactos, sem terem sido modificados - seja por um ator malicioso, perda de dados ou por algum outro fator. 

\textbf{A disponibilidade} se refere à acessibilidade dos dados quando necessário.


Examinaremos as várias formas de criptografia digital e como elas podem nos ajudar a alcançar os três objetivos listados acima. Quando falamos de criptografia digital, geralmente nos referimos a uma das seguintes criptografias, que serão melhor explicadas e exemplificadas na própria sessão.

\begin{enumerate}
  \item Criptografia simétrica
  \item Criptografia assimétrica
  \item Funções de hash
\end{enumerate}

Esses conceitos serão explicados e exemplificados na próxima secção.

É temido que os computadores quânticos sejam capazes de decifrar certos códigos usados para enviar mensagens seguras. Os códigos em questão criptografam dados usando funções matemáticas de ``trapdoor'' que funcionam facilmente em uma direção, mas não em outra. Isso facilita a criptografia de dados, mas a decodificação é extremamente difícil sem a ajuda de uma chave especial.

Esses sistemas de criptografia nunca foram inquebráveis, sua segurança se baseia na enorme quantidade de tempo que um computador clássico levaria para fazer o trabalho. Os métodos modernos de criptografia são projetados especificamente para que a decodificação demore tanto tempo de forma a serem praticamente inquebráveis.

No entanto, os computadores quânticos mudaram esse pensamento. Essas máquinas são muito mais poderosas que os computadores clássicos, tornando possível o rompimento desses códigos com facilidade, já que realizam contas matemáticas de forma mais eficiente e significativamente mais rápido que os computadores clássicos.

\subsection{Conceitos básicos de criptografia}
Antes de mergulharmos nisso: o que exatamente queremos dizer com "criptografia"? Criptografar e descriptografar são normalmente usadas para significar criptografia e decifração, respectivamente; Para simplificar, criptografar uma mensagem significa torná-la ilegível para partes não autorizadas usando uma cifra (o método específico para fazer isso). Descriptografar a mensagem significa reverter o processo e tornar os dados legíveis mais uma vez.

\subsubsection{Criptografia simétrica}
Para criptografar e descriptografar corretamente nossos dados, precisamos dos dados e de uma chave (que determina a saída da nossa cifra).
Com a criptografia simétrica, a chave usada para criptografar e descriptografar dados é a mesma.

\subsubsection{Criptografia assimétrica}
O problema da criptografia simétrica é o seguinte: E se eu precisar enviar dados com segurança em um ambiente hostil, como a Internet? Se a mesma chave for usada para criptografar e descriptografar dados, primeiro eu precisaria enviar a chave de descriptografia para estabelecer uma conexão segura. Mas isso significa que estou enviando a chave por uma conexão insegura, o que significa que a chave pode ser interceptada e usada por terceiros! Como contornar isso? 

Para exemplificar, usaremos um cadeado que possui três estados: A (bloqueado), B (desbloqueado) e C (bloqueado).
E tem duas chaves distintas. A primeiro pode girar apenas no sentido horário (de A a B a C) e a segunda pode girar apenas no sentido anti-horário (de C a B a A).

Ao criptografar uma mensagem, o usuário pega a primeira chave e guarda para si mesmo. Essa chave, sua chave "privada" - porque apenas ele a possui.

A segunda chave, sua chave “pública”: Pode ser distribuida para qualquer pessoa. Assim, o usuário tem sua chave privada que pode mudar de A para B para C. E todo os outros tem sua chave pública que pode mudar de C para B para A.

Colocando isso em pratica, imagine que você queira enviar um documento privado para o usuario. Você coloca o documento na caixa e usa uma cópia da chave pública dele para bloqueá-lo. Lembre-se de que a chave pública dele gira apenas no sentido anti-horário, e você a coloca na posição A. Agora a caixa está bloqueada. A única chave que pode passar de A para B é a chave privada, a que ele guardou para si.

\subsubsection{Funções de hash}
Uma função de hash, diferente da criptografia simétrica / assimétrica, é uma função unidirecional. Você pode criar um hash a partir de alguns dados, mas não há como reverter o processo. Como tal, não é uma maneira útil de armazenar dados, mas é uma maneira útil de verificar a integridade de alguns dados.
Uma função de hash recebe alguns dados como entrada e gera uma string aparentemente aleatória (mas nem tanto) que sempre terá o mesmo comprimento. Uma função de hash ideal cria valores exclusivos para diferentes entradas. A mesma entrada exata sempre produzirá exatamente o mesmo hash - e é por isso que podemos usá-la para verificar a integridade dos dados.

\subsection{Rivest Shamir Adleman – RSA Criando chave publica e privada}

% \begin{algorithm}[H]
%   \SetAlgoLined
%   \KwResult{Write here the result }
%    $p\prime$
%    \newline
%    $q\prime$
%    \newline
%    $n = p * q$
%    \newline
%    $\Phi(n)=(p-1)(q-1)$
%    \newline
%    initialization\;
%    \While{While condition}{
%     instructions\;
%     \eIf{condition}{
%      instructions1\;
%      instructions2\;
%      }{
%      instructions3\;
%     }
%    }
%    \caption{How to write algorithms}
%   \end{algorithm}

\vspace{1cm}
\begin{longtable}{ |p{6cm}|| p{8cm}|  }
  \hline
  \multicolumn{2}{|c|}{RSA – Cryptosystem} \\
  \hline
    Descrição & matemática\\
  \hline
    Escolher dois números primos & 
    \[p=2\] \[q=7\]\\
  \hline
    Produto dos números escolhidos & 
    \[n=14\]\\
  \hline
    função Phi 
    \[\Phi(n)=(p-1)(q-1)\] & 
    \[\Phi(14)=(7-1)(2-1)\]
    \[\Phi(14)=(6)(1)\]
    \[\Phi(14)=6\]
    \[Coprimos\: de\: 14:\: 1, 3, 5, 9, 11, 13\]\\
  \hline
    Escolher o numero e `encryption'
    \begin{itemize}
      \item $1 < e < \Phi(n)$
      \item $coprimo\: de: n\: e\: \Phi(n)$
    \end{itemize} &
    \[Coprimos\: de\: 14:\: 1, 3, 5, 9, 11, 13\]
    \[Coprimos\: de\: 6:\: 1, 5\]
    \[Menor\: que\: 6\: e\: coprimo\: de\: 14\: e\: de\: 6: 5\]
    \[e = 5\]\\
  \hline
    Chave publica & 
    \[(5, 14)\]\\
  \hline
  Escolher o numero d `decryption'
    \begin{itemize}
      \item $d * e (mod \Phi(n)) = 1$
    \end{itemize} & 
    \[d * 5 (mod 6) = 1\]
    \[5*d = 5, 10, 15, 20, 25 \dots 55\]
    \[5*1 (mod 6) = 5\]
    \[5*2 (mod 6) = 4\]
    \[5*3 (mod 6) = 3\]
    \[5*4 (mod 6) = 2\]
    \[5*5 (mod 6) = 1\]
    \[5*6 (mod 6) = 0\]
    \[ \dots \]
    \[5*11 (mod 6) = 1\] \\
  \hline
  Chave privada & 
  \[(11, 14)\]\\
\hline
\end{longtable}

\vspace{1cm}
\begin{longtable}{ |p{6cm}|| p{8cm}|  }
  \hline
  \multicolumn{2}{|c|}{RSA – Cryptosystem} \\
  \hline
    Descrição & matemática\\
  \hline
    Criptografar \newline
    Chave publica: $(5, 14)$ \newline
    mensagem: ``B'' 
    \[A \to 1\]
    \[B \to 2\]
    \[C \to 3\]
    \[D \to 4\] & 
    \[2^5(mod 14)\]
    \[32(mod 14)\]
    \[4(mod 14)\]
    Mensagem criptografada: 4 $\to$ D\\
  \hline
    Descriptografar \newline
    Chave privada: $(11, 14)$ \newline
    mensagem: ``D'' 
    \[A \to 1\]
    \[B \to 2\]
    \[C \to 3\]
    \[D \to 4\] & 
    \[4^{11}(mod 14)\]
    \[4194304(mod 14)\]
    \[2(mod 14)\]
    Mensagem original: 2 $\to$ B\\
  \hline
\end{longtable}

\subsection{Criptografia aplicada computação clássica}
\subsection{Criptografia aplicada computação quântica}